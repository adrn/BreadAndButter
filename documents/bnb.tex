% This file is part of the BreadAndButter project.
% Copyright 2013 David W. Hogg (NYU) and any other authors.

\documentclass[letterpaper,12pt,preprint]{aastex}

\newcommand{\project}[1]{\textsl{#1}}
\newcommand{\gaia}{\project{Gaia}}
\newcommand{\given}{\,|\,}
\newcommand{\transpose}[1]{{#1}^{\mathsf{T}}}
\newcommand{\inverse}[1]{{#1}^{-1}}

\begin{document}

\title{Inferring the gravitational potential of the Milky Way with measurements of \emph{just two stars}}
\author{DWH, APW, KVJ, others}

\begin{abstract}
Informative phase-space structures, and in particular those created by
disrupting stellar systems or revealed by chemical tags, may deliver
very precise measures of the gravitational potential in the Milky Way
Halo.  Here we show that even a single pair of stars---two stars that
are known (for some non-kinematic reason) to be likely to be
associated with one another at birth---could provide a significant
constraint on the potential.  The inference is based on a
probabilistic generative model---a likelihood function and priors over
nuisance parameters---that evolves a past putative origin disruption
event forward in time to the present day.  The time, six-volume, and
phase-space location of the origin event are nuisance parameters in
the model and marginalized away.  The method makes no assumption of
integrability, works with finite or even large observational
uncertainties, does not require all dimensions of phase space to be
observed, handles non-zero probability that the two stars are not in
fact associated, and generalizes naturally to larger numbers of stars
and multiple independent structures.  Applications to the GD-1 cold
stellar stream and current surveys for RR Lyrae stars in the Halo are
discussed.
\end{abstract}

\keywords{
  this ---
  that ---
  Milky Way
}

\section{Introduction}

[Cold streams contain tons of information.]

[GD-1 was fit as if it highlighted an orbit.  That is known to be
  wrong.]

[In the halo, any phase-space structure might be long-lived.  Chemical
  tagging could in principle illuminate it.]

[Chemical tagging will always only give probabilistic information.
  But what if it delivers somewhat confident information about small
  stellar families in the halo?]

[Actions, angles, integrable, not, etc.  Can integrate in any
  potential, even a time-varying one.  APW \& KVJ have used this to
  ``run the clock back''.  Here we only let ourselves \emph{generate
    the data}.  Why is an objective based on past collisions not the
  same as assuming that the stars are on the same orbit?]

[Sometimes stellar samples might get extremely small.  For example,
  \gaia\ might find some very low-mass cold stellar streams.  For
  another, there might only be a coupld RRL stars in a particular
  stream, but those RRL stars might be individually extremely well
  measured in six-space.]

[Obviously a small stellar sample can only tell you a small number of
  things about the Milky Way potential.  The idea is---in the long
  run---to construct likelihood functions for many small stellar
  samples and in the end do inference by multiplying them together.
  Duh!]

\section{The model}

The first idea is that there are $N\geq 2$ stars $n$ with \emph{true}
phase-space 6-vector positions $X_n$.  Each 6-vector position is a
column vector containing the position 3-vector and the velocity
3-vector.  In reality, these phase-space positions are not directly
observed but rather there are some data $D_n$ for each star, and a
likelihood function $p(D_n\given X_n)$ that expresses the
probabilistic relationship between the data and the true 6-vector
position.  This description is agnostic about what these data might
be, but we are imagining celestial positions, a parallax, proper
motions, and a radial velocity, or some subset of those, all with
(fairly well understood) uncertainties.  In some cases the distance
information might come from photometry rather than astrometric
parallax.

The second idea is that these $N\geq 2$ stars $n$ have some finite
prior probabilities $P_{nk}$ that they were once members of a compact
phase-space structure labeled $k$ in the past.  At first we are
thinking only about the existence of a single structure $k$, but in
the end we can generalize to $K$ structures.  Technically these
$P_{nk}$ ought to be the probabilities based on any data available
prior to, or independently of, the phase-space information used in
this experiment.  So, for example, if the stars are chosen from a box
in chemical abundances, the probabilities $P_{nk}$ would be related to
the probabilities that they obtain their similar chemical-abundances
pattern inferences just by chance, without any relationship in their
origin.  In many cases, the $P_{nk}$ are not known or hard to
estimate; in these cases they can in fact be inferred (as we will
demonstrate below).

We want to perform inference \emph{without} building and
running---within an MCMC chain---a fully dynamical model of a
disrupting satellite in the Milky Way.  We model the progenitor $k$ as
a Gaissian blob in six-space composed of a spherical Gaussian of
isotropic variance $\sigma_{kx}^2$ in the three positional dimensions
and a spherical Gaussian of isotropic variance $\sigma_{kv}^2$ in the
three velocity dimensions.  These two variances are combined into a
diagonal $6\times 6$ positive variance tensor $\Sigma_k$; that is, the
variance tensor $\Sigma_k$ is specified with two scalar variances,
$\sigma_{kx}^2$ and $\sigma_{kv}^2$.  This Gaussian blob in 6-space
itself orbits on a gravitational orbit in the potential; that is it's
mean 6-vector position in phase space as a function of time is
determined by taking some 6-vector initial conditions $X_k$ and
integrating through time along a gravitational orbit.  That is, in
addition to the variance tensor $\Sigma_k$, the progenitor has a
6-vector position $X(t\given X_k)$, parameterized by time $t$ and
initial conditions $X_k$.  For definiteness, we will define the
present day to be $t=0$ (ignoring light-travel times, of course), such
that all measurements of stars $D_n$ are made at $t=0$ and the
progenitor initial conditions are set at $t=0$.

\section{Experiments}

[Each experiment requires a description of how the fake data were
  generated, what the likelihood function is, and posterior potential
  information obtained.]

[basic experiment: $N=2$, $K=1$, high $P_{nk}$, all dimensions
  measured at best-present-day quality]

[\gaia\ quality]

[low $P_{nk}$, MW bg model]

[unknown $P_{nk}$]

[missing data]

[$N>2$]

[$K>1$]

[Can we do a real experiment with two real RRL stars?]

\section{Discussion}

[What's unrealistic about what we have done?  The kinematic model of
  the progenitor is exceedingly simplistic; why are we okay with it
  nonetheless but what could we do to improve it?  Related to this,
  the progenitor doesn't shrink with time (as it should).  What's up
  with that?]

[Issues about computation and scaling.]

[Don't be afraid of nuisance parameters!]

[Don't eschew low-dimensional constraints on the potential!]

\acknowledgements
Binney, Rix, Sanders

\end{document}
